\chapter{Appendix title}
\label{appendix:appendix_1} 


\lipsum[2-4]

To estimate the location of a step change, a Bayesian change point detection algorithm based on \cite{Ruanaidh1996} and \cite{adams2007bayesian} is used in the thesis. Given a data sequence $x$ of $N$ samples with Gaussian noise added:

\begin{equation}
    \centering
      x_i=\begin{cases}
          \mu_1 + \epsilon_i, & \text{if $i<m$}.\\
          \mu_2 + \epsilon_i, & \text{otherwise}.
          \end{cases}
    \label{eq:app:cp:5.1}
\end{equation}

where the noise samples $\epsilon_i$ are assumed to be independent and $m$ is the step change location. The likelihood of the data is given by the joint probability of the noise samples $\epsilon_i$: 

\begin{equation}
    \centering
    P(x|\{\mu_1\mu_2\sigma m\}) = \prod\limits_{i=1}^N P(\epsilon_i)
    \label{eq:app:cp:5.2}
\end{equation}

where $\sigma$ is the standard deviation of the Gaussian noise;  $\mu_1, \mu_2$ and $\sigma m$ are the known time series parameters. The probability density function for the noise samples is defined by:

\begin{equation}
    \centering
    P(\epsilon) = \frac{1}{\sigma \sqrt{2 \pi} }  e ^{ - \frac{ (\epsilon - \mu)^2 } {2\sigma^2} }
    \label{eq:app:cp:noise_pdf}
\end{equation}

\lipsum[2-4]